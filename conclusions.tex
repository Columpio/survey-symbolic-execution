% !TEX root = main.tex

\section{Conclusions}
\label{se:conclusions}

Techniques for symbolic execution have evolved significantly in the last decade, leading to major practical breakthroughs. In 2016, DARPA has challenged the global innovation community with a \$2M prize to build a computer that can hack and patch unknown software with no one at the keyboard. The winner, {\sc Mayhem}~\cite{MAYHEM-SP12}, was also the first autonomous computer system to play the Capture-The-Flag contest at the DEF CON 24 hacker convention\footnote{\url{https://www.defcon.org/html/defcon-24/dc-24-ctf.html}.}. The event demonstrated that tools for automatic exploit detection based on symbolic execution can be competitive with human experts, paving the road to unprecedented applications and the rise of start-ups that have the potential to shape software security and reliability in the next decades. 

This survey has discussed some of the key aspects and challenges of symbolic execution, presenting them for a broad audience. To explain the basic design principles of symbolic executors and the main optimization techniques, we have focused on single-threaded applications with integer arithmetic. Symbolic execution of multi-threaded programs is treated, e.g., in~\cite{KPV-TACAS03,SA-HVC06,CLOUD9-EUROSYS11,FHR-ESEC13,BGC-OOPSLA14,GKW-ESEC15}, while techniques for programs that manipulate floating point data are addressed in, e.g., \cite{M-STVR01,BGM-STVR06,LTH-ICTSS10,CCK-EUROSYS11,BVL-POPL13,CCK-TSE14,RPW-SIGSOFT15}.

We hope that this survey will help non-experts to grasp the key inventions in the exciting line of research of symbolic execution, inspiring further work and new ideas.
