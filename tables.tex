% !TEX root = appendix.tex

\subsection{Additional Tables}

\subsubsection{Tools}

\begin{table}[t]
  \centering
  \begin{adjustbox}{width=0.98\columnwidth} % TODO was 1
  %\begin{small}
  \begin{tabular}{| l || c || l |}
    \hline      
    {\bf Symbolic engine} & {\bf References} & {\bf Project URL} (last retrieved: August 2016)  \\ \hline\hline
    
    % CNC is not a symbolic engine but it uses constrained solver
    %{\sc Check 'n' Crash} & \cite{CS-ICSE05} & \url{http://ranger.uta.edu/~csallner/cnc/}\\
    
    {\sc CUTE} & \cite{CUTE-FSE05} & -- \\
    {\sc DART} & \cite{DART-PLDI05} & -- \\
    {\sc jCUTE} & \cite{SA-CAV06} & \url{https://github.com/osl/jcute} \\ % : Java Concolic Unit Testing Engine
    {\sc KLEE} & \cite{EXE-CCS06,KLEE-OSDI08} & \url{https://klee.github.io/} \\ % : a LLVM Execution Engine
    {\sc SAGE} & \cite{SAGE-NDSS08,EGL-ISSTA09} & -- \\
    {\sc BitBlaze} & \cite{BITBLAZE-ICISS08} & \url{http://bitblaze.cs.berkeley.edu/} \\ % , BHK-TR07
    {\sc CREST} & \cite{CREST-ASE08} & \url{https://github.com/jburnim/crest} \\ % : a concolic test generation tool for C
    {\sc PEX} & \cite{PEX-TAP08} & \url{http://research.microsoft.com/en-us/projects/pex/} \\
    {\sc Rubyx} & \cite{CF-CCS10} & -- \\
    {\sc Java PathFinder} & \cite{PATHFINDER-ASE10} & \url{http://babelfish.arc.nasa.gov/trac/jpf}\\
    {\sc Otter} & \cite{RSM-ICSE10} & \url{https://bitbucket.org/khooyp/otter/} \\
    {\sc BAP} & \cite{BAP-CAV11} & \url{https://github.com/BinaryAnalysisPlatform/bap} \\
    {\sc Cloud9} & \cite{CLOUD9-EUROSYS11} & \url{http://cloud9.epfl.ch/} \\
    {\sc Mayhem} & \cite{MAYHEM-SP12} & -- \\
    {\sc SymDroid} & \cite{JMF-TECH12} & -- \\
    {\sc \stwoe} & \cite{CKC-TOCS12} & \url{http://s2e.epfl.ch/} \\
    {\sc FuzzBALL} & \cite{MMP-ASPLOS12,FUZZBALL-ESORICS13} & \url{http://bitblaze.cs.berkeley.edu/fuzzball.html} \\
    {\sc Jalangi} & \cite{SKB-FSE13} & \url{https://github.com/Samsung/jalangi2} \\
    {\sc Pathgrind} & \cite{S-ICSE04} & \url{https://github.com/codelion/pathgrind} \\
    {\sc Kite} & \cite{V-THESIS14} & \url{http://www.cs.ubc.ca/labs/isd/Projects/Kite} \\
    {\sc SymJS} & \cite{LAG-FSE14} & -- \\
    {\sc CIVL} & \cite{CIVL-SC15} & \url{http://vsl.cis.udel.edu/civl/}\\ % : The Concurrency Intermediate Verification Language 
    {\sc KeY} & \cite{HBR-RV14} & \url{http://www.key-project.org/} \\
    {\sc Angr} & \cite{FIRMALICE-NDSS15,ANGR-SSP16} & \url{http://angr.io/} \\
    {\sc Triton} & \cite{TRITON-SSTIC15} & \url{http://triton.quarkslab.com/} \\
    {\sc PyExZ3} & \cite{BD-TECH15} & \url{https://github.com/thomasjball/PyExZ3} \\
    {\sc JDart} & \cite{JDART-TACAS16} & \url{https://github.com/psycopaths/jdart} \\

    {\sc CATG} & -- & \url{https://github.com/ksen007/janala2} \\
    {\sc PySymEmu} & -- & \url{https://github.com/feliam/pysymemu/} \\
    {\sc Miasm} & -- & \url{https://github.com/cea-sec/miasm} \\
    
    \hline  
  \end{tabular}
  %\end{small}
  \end{adjustbox}
  \caption{Selection of symbolic execution engines, along with their reference article(s) and software project web site (if any).}
  \label{tab:symbolic-engines}
  \vspace{-3mm} % TODO
\end{table}

Table~\ref{tab:symbolic-engines}\mynote{check URLs} lists a number of symbolic execution engines that have worked as incubators for several of the techniques surveyed in this article. The novel contributions introduced by tools that represented milestones in the area are described in the appropriate sections throughout the article.

\subsubsection{Path Selection Heuristics}

\begin{figure}[t]
  \centering
  \begin{adjustbox}{width=0.84\columnwidth} % TODO was 1; with 0.88 the last paragraph will fit
  \begin{small}
  \begin{tabular}{| l || l |}
    \hline      
    {\bf Heuristic} & {\bf Goal} \\ \hline\hline
    \multirow{2}*{BFS} & {\em Maximize coverage} \\ & \cite{CKC-TOCS12,PEX-TAP08} \\\hline
    \multirow{3}*{DFS} & {\em Exhaust paths, minimize memory usage} \\ & \cite{EXE-CCS06,CKC-TOCS12}\\ & \cite{PEX-TAP08,DART-PLDI05} \\\hline
    \multirow{2}*{Random path selection} & {\em Randomly pick a path with probability based on its length} \\ & \cite{KLEE-OSDI08} \\\hline
    %low-covered code & prioritize paths that execute low-covered code  & \cite{EXE-CCS06} \\
    \multirow{3}*{Code coverage search} & {\em Prioritize paths that may explore unexplored code} \\ & \cite{EXE-CCS06,KLEE-OSDI08,MAYHEM-SP12}\\ & \cite{CKC-TOCS12,GV-ISSTA02} \\\hline
    \multirow{2}*{Buggy-path-first} & {\em Prioritize bug-friendly path} \\ & \cite{AEG-NDSS11} \\\hline
    \multirow{2}*{Loop exhaustion} & {\em Fully explore specific loops} \\ & \cite{AEG-NDSS11} \\\hline
    \multirow{2}*{Symbolic instruction pointers} & {\em Prioritize paths with symbolic instruction pointers} \\ & \cite{MAYHEM-SP12} \\\hline
    \multirow{2}*{Symbolic memory accesses} & {\em Prioritize paths with symbolic memory accesses} \\ & \cite{MAYHEM-SP12} \\ \hline
    \multirow{2}*{Fitness function} & {\em Prioritize paths based on a fitness function} \\ & \cite{XTD-DSN09,CS-CACM13,XTD-DSN09} \\ \hline
    \multirow{3}*{Subpath-guided search} & {\em Use frequency distributions of explored subpaths to prioritize}\\ & {\em less covered parts of a program} \\ & \cite{LZL-OOPSLA13} \\
    %kill path & filter uninteresting path & \cite{CKC-TOCS12} \\
    \hline  
  \end{tabular}
  \end{small}
  \end{adjustbox}
  \caption{Common path selection heuristics discussed in the literature.}
  \label{tab:heuristics}
\end{figure}
