% !TEX root = paper.tex

\subsection{Path Explosion}
\label{ss:path-explosion}

%Language constructs such as loops might exponentially increase the number of execution states. It is thus unlikely that a symbolic execution engine can exhaustively explore all the possible states within a reasonable amount of time. In practice, heuristics are used to guide exploration and prioritize certain states first (e.g., to maximize code coverage). In addition, symbolic engines can implement efficient mechanisms for evaluating multiple states in parallel without running out of resources.

One of the main challenges of symbolic execution is the path explosion problem. As symbolic execution may fork off a new execution engine instance at every branch, the total number of executors may be exponential in the number of branches in the program. This impacts both time and space, since a symbolic executor may need to keep track of an exponential number of pending branches to be explored.

It is thus unlikely that a symbolic execution engine can exhaustively explore all the possible states within a reasonable amount of time. In practice, a common approach is to compute an under-approximation of the analysis that only explores a relevant subset of the state space.

%In practice, several heuristics must be exploited to prioritize evaluation of some states, hoping to still be able to spot interesting things. Moreover, the symbolic execution engine should include efficient mechanism for efficiently evaluating in parallel different execution states without running out of computational resources.
%\mynote{I: We could add the following two examples to explain the causes of path explosion (Vechev)}

% ---------------------------------------------------------------------------------------------------
\myparagraph{Pruning unrealizable paths}
A first natural technique for reducing the path space is invoking the constraint solver at each branch, pruning branches that are not realizable. Indeed, if the constraint solver is able to prove that the logical formula given by the path constraints of a branch is not satisfiable, then there exists no assignment of the program input values that would drive a real execution toward that path. For this reason, the symbolic engine can safely discard the path involving that branch without affecting soundness of the approach. 

% There are several strategies for selecting or generating the next path to be explored. We now briefly overview some of the most interesting techniques that have been shown to be effective in the literature. %in prior works.

\myparagraph{Search heuristics} Another common approach is to limit the amount of resources symbolic execution is allowed to use. For instance, the computation may time out after a certain amount of time. Since only a fraction of paths may be explored, the search should be prioritized by looking at the most promising paths first. Given a set of unexplored paths, a {\em search heuristic} should select the most promising path to explore. Many works have introduced novel search strategies, showing their effectiveness in specific application contexts. These heuristics have often been tailored to help the symbolic engine achieve a specific goal (e.g., overflow detection).

%Finding a universally optimal strategy for prioritizing path exploration remains an open problem. Table~\ref{tab:heuristics} provides a sample of prominent search heuristics discussed in prior works. 

The most common strategies are {\em depth-first search} (DFS) and {\em breadth-first search} (BFS). DFS continuously expands a path as much as possible, before backtracking to the deepest unexplored branch. BFS explores all unexplored paths in parallel, repeatedly expanding each of them by a fixed slice. DFS is often adopted for minimizing the memory usage of the symbolic engine. Unfortunately, paths containing loops and recursive calls can easily stall the symbolic engine. For this reason, some tools prefer prioritizing paths using BFS.

%Another popular strategy is {\em random path selection} that, as its name would suggest, randomly picks a path for exploration. This heuristic has been refined in several variants. For instance, {\sc KLEE}~\cite{KLEE-OSDI08} assigns probabilities to paths based on their length and on the branch arity. Namely, it favors paths that have been explored fewer times, preventing starvation caused by loops and other path explosion factors.

% TODO add this one maybe?
%Several works, such as {\sc EXE}~\cite{EXE-CCS06}, {\sc KLEE}~\cite{KLEE-OSDI08}, {\sc Mayhem}~\cite{MAYHEM-SP12}, and {\sc \stwoe}~\cite{CKC-TOCS12}, have discussed heuristics aimed at maximizing code coverage. For instance, the {\em coverage optimize search} discussed in {\sc KLEE}~\cite{KLEE-OSDI08} computes for each state a weight, which is later used to randomly select states. The weight is obtained by considering how far the nearest uncovered instruction is, whether new code was recently covered by the state, and the state's call stack.

%Of a similar flavor is the heuristic proposed in~\cite{LZL-OOPSLA13}, called {\em subpath-guided search}, which attempts to explore {\it less traveled} parts of a program by selecting the subpath of the control flow graph that has been explored fewer times. This is achieved by maintaining a frequency distribution of explored subpaths, where a subpath is defined as a consecutive subsequence of length $n$ from a complete path. Interestingly, the value $n$ plays a crucial role with respect to the code coverage achieved by a symbolic engine using this heuristic and no specific value has been shown to be universally optimal.

Other search heuristics try to prioritize paths likely leading to states that are {\em interesting} according to some goal. For instance, the {\em buggy-path first} strategy in {\sc AEG}~\cite{AEG-NDSS11} picks paths whose past states have contained small but unexploitable bugs. The intuition is that if a path contains some small errors, it is likely that it has not been properly tested. There is thus a good chance that future states may contain interesting, and hopefully exploitable, bugs.
%Similarly, the {\em loop exhaustion} strategy discussed in {\sc AEG}~\cite{AEG-NDSS11} explores paths that visit loops. This approach is inspired by the practical observation that common programming mistakes in loops may lead to buffer overflows or other memory-related errors.
In order to find exploitable bugs, {\sc Mayhem}~\cite{MAYHEM-SP12} instead gives priority to paths where symbolic memory accesses are identified or symbolic instruction pointers are detected.


\myparagraph{State merging}
Another way to limit resource usage and optimize exploration in a symbolic executor is to introduce merged states characterizing sets of execution paths. In other words, a merged state will be described by a formula that represents the disjunction of the formulas that would have described the individual states if they were kept separate. 
% ~\cite{G-POPL07,HSS-RV09}
Early works have shown that merging techniques effectively decrease the number of paths to explore, but also put a burden on constraints solvers, which typically encounter difficulties when dealing with disjunction. \cite{KKB-PLDI12} provides an excellent discussion of the design space of state merging techniques. In particular, a symbolic engine can adopt heuristics to identify and perform only those merges that can speed the exploration process up. Indeed, generating larger symbolic expressions and possibly extra solvers invocations can outweigh the benefit of having fewer states, resulting into poorer performance.%, leading to poorer overall performance~\cite{HSS-RV09,KKB-PLDI12}.