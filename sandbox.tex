%%% MEMORY MODELS
\vspace{2em}\mynote{[D] Previous text starts here}In other words, a symbolic engine may see symbols as distinct objects (i.e., each symbol is a distinct array of a specific size) or as pointers to a flat memory (i.e., index-based memory). Although the latter approach may seem more natural since it is akin to a concrete execution model, the former has been proved to be very effective in many scenarios since the symbolic constraints generated when using this approach are {\em easier} to parse for some solvers.

% ---------------------------------------------------------------------------------------------------
\subsection{Index-based memory}

\begin{itemize}
  \item memory is a map $\pi : I \to E$ from 32-bit indexes ($i \in I$) to expressions ($e \in E$)
  \item load expressions $e = load(\pi, i)$: $i$ indixes $\pi$ and the loaded value $e$ represents the contents of the $i$-th memory cell
  \item store expressions $store(\pi, i, e)$: a new memory $\pi'$ where $i$ is mapped to $e$, i.e., $\pi' = \pi[i \gets e]$
\end{itemize}
When using this memory model, handling of arbitrary symbolic indexes is notoriously hard, since a symbolic index may reference any cell in memory. Two approaches can be pursued: (a) concretization of the index where only a single value is evaluated for the index (see Section~\ref{concolic-execution} for more details), (b) fully symbolic memory where any possible value for the index is evaluated (e.g., \cite{BAP-CAV11}). The former is often excessively limiting (many paths are pruned away), while the latter is hard to make scalable. To overcome the limitations of both approaches, \cite{MAYHEM-SP12} models memory {\em partially}: symbolic writes are always concretized, while symbolic reads are allowed to be modeled symbolically.

\paragraph{Memory objects} Whenever there is a symbolic read, an immutable memory object $\mathcal{M}$ is generated:  it contains all values that could be accessed by the index, i.e., $\mathcal{M}$ is a partial snapshot of the global memory $\pi$. In practice, \cite{MAYHEM-SP12} reasons on $\mathcal{M}[i]$ instead of reasoning on $\pi[i]$, since the former is typically smaller than the latter.

\paragraph{Memory object bounds resolution}\mynote{Move paragraphs to Section~\ref{constraint-optimizations}?} \cite{MAYHEM-SP12} trades accuracy with scalability by resolving the bounds $[\mathcal{L}, \mathcal{U}]$ of the memory region, where $\mathcal{L}$ is lower and $\mathcal{U}$ is the upper bound for the index $i$. Initially, $\mathcal{L} \in [0, 2^{32}-1]$. The solver is used to test if $i < \frac{2^{32}-1}{2}$ makes the path constrains unsatisfiable. If satisfiable then $\mathcal{L} \in [0, \frac{2^{32}-1}{2}]$, otherwise $\mathcal{L} \in [\frac{2^{32}-1}{2}, 2^{32}-1]$. The same strategy is repeated as much as possible. This is akin to a binary search algorithm. Whenever the bounds become reasonable, a memory object is generated. Unfortunately, there several cases where the bounds cannot be narrowed down drastically and thus other techniques can be used:
\begin{itemize}
  \item {\em value set analysis (VSA)}
  \item {\em refinement cache}
  \item {\em lemma cache}
  \item {\em index search trees (IST)}
  \item {\em bucketization with linear functions}
\end{itemize}
Whenever the size of the memory object exceeds a threshold (e.g., $|\mathcal{M}| \geq 1024$), then~\cite{MAYHEM-SP12} concretizes the index. However, instead of choosing a random value for the index, it tries to assign some {\em interesting} value (e.g., test it if it makes sense for it to be equal to an invalid memory address which can be exploited for security purposes).

% ---------------------------------------------------------------------------------------------------
\subsection{Object-based memory}

\cite{STP-TR07} is a decision procedure for bitvectors and arrays. Memory is seen as untyped bytes. Three data types are available:
\begin{itemize}
  \item {\em booleans}
  \item {\em bitvectors}: a fixed-length sequence of bits
  \item {\em arrays of bitvectors}
\end{itemize}
Most linear and non-linear operations are mapped to bitvector constraints. Conditional branches are transformaed into {\em multiplexers}, which are similar to C ternary operator. Bitvector operations are translated into operations to individual bits. Floating-points data types are not supported. Expresions types:
\begin{itemize}
  \item {\em formulas}, which have boolean values. They are converted into DAGs of single bit operations, where expressions with identical structure are represented uniquely (an hash table is maintain to track of existing expressions and lookups are performed on it when a new expression is created).
  \item {\em terms}, which have bitvectors values. They are converted into vectors of boolean formulas consisting entirely of single bit operations.
\end{itemize}

\paragraph{Mapping C code to STP primitives} Each C data block is represented symbolically as an array of 8-bit bitvectors. Typed operations in C generated constraints on the symbolic data (i.e., typeness is not actually known to~\cite{STP-TR07}). A mapping table is maintained to track symbolic data:
\begin{itemize}
  \item each input array $b$ is associated to a symbolic identically-size array $b_{sym}$ (i.e., the address of the C variable $b$ is mapped to the symbolic array $b_{sym}$ that is composed by $|b|$ 8-bit elements)
  \item $v = e$: an assignment expression, where $e$ is an expression that involves one or more symbolic data, adds a mapping between the address of $v$ and the generated symbolic expression $e_{sym}$. If $v$ is overwritten with a constant value or deallocated then this mapping is removed.
  \item $b[e]$: $b_{sym}$ is allocated, initializing it with the (constant) contents of $b$ (only if $b$ is actually initialized by a previous C statement). A mapping is added until $b$ is not deallocated.
\end{itemize}

\paragraph{Expression evaluation} An expression $e$ that contains some symbolic data can be seen as:
\[ l_1~~op_1~~l_2~~op_2~~l_3~~... \]
Each $l_i$ is evaluated in the following way:
\begin{itemize}
  \item if $l_i$ is concrete: the concrete value is used in the expression (e.g., if 4-byte $b$ is equal to 4 then the constant $000000...0100$ is used)
  \item otherwise: a concatenation of all the symbolic bytes of $l_i$ is used (e.g., $b_{sym}[0] @b_{sym}[1] @ $ $b_{sym}[2] @ b_{sym}[3] @ ...$)
\end{itemize}
Pointers are seen as array reference at some offset. This means that allocation sites as well as pointer arithmetic expressions must be instrumented in order to track where a pointer can point to. Notice that double dereferences ({\tt **p}) force~\cite{STP-TR07} to concretize the first dereference ({\tt *p}). 

\paragraph{Fast array transformations}\mynote{Move paragraphs to Section~\ref{constraint-optimizations}?} Some transformations are needed to make\cite{STP-TR07} reason only on a purely functional language. In particular:
\begin{itemize}
  \item {\em read-over-write}: eliminates all write operations
    \[ read(write(A, i, v), j) \implies ite(i = j, v, read(A, j)) \]
    where $ite(a,b,c)$ is a ternary operator (i.e., if $a$ then $b$ else $c$). Notice that a write of a location without a subsequent read of the same location can be ignored.
  \item {\em read elimination}:
    \[ (read(A, i) = e_1) \wedge (read(A, j) = e_2) \]
    will be transformed in:
    \[ (v_1 = e_1) \wedge (v_2 = e_2) \wedge (i=j \implies v_1 = v_2) \]
  \item {\em array substitution optimization}
  \item {\em array-based refinement}
  \item {\em simplifications}: boolean or mathematical identities
\end{itemize} 

%%% COMPLEX OBJECTS

It is common to use {\em lazy initialization} for handling complex memory objects. Notice that if lazy initialization is not used by the symbolic engine, then it is likely that it has to treat complex objects completely symbolically, requiring a constraint solver that is able to solve the resulting constraints. In other words, the solver must support some form of theory of data structures or arrays.

\paragraph{Recursive data structures} Lazy initialization over recursive data structure works~\cite{PV-JSTTT09} as follows:
\begin{itemize}
  \item whenever an instance method is called over a symbolic object, the object is created with uninitialized fields
  \item whenever an uninitialized field of a non-native type is accessed, the symbolic engine non deterministically initialize it with: (a) a {\tt null} reference, (b) a reference to a previously created object, (c) a reference to a new object with uninitialized fields. 
  \item whenever an uninitialized field of a native type\mynote{this is my guess, it's not clear explained in~\cite{PV-JSTTT09} - [D] correct, see KPV-TACAS03}) is accessed, a new symbolic value is introduced. 
\end{itemize}

\mynote{Merge discussion of {\em lazy initialization} done in Section~\ref{under-constrinained}.}

Notice that if some input preconditions~\ref{precontioned-symbolic-execution} has been set, lazy evaluation should consider and exploit them. For instance, the object may have been marked as acyclic and thus fewer possible alternatives should be considered when initializing a uninitialized field. Further complication~\cite{PV-JSTTT09} may arise for programs that perform destructive updates. 

\paragraph{Native code} Several programming languages provide data types that can be very complex. For instance, Java provides the {\tt String} type. Prior works (see, e.g.,~\cite{SHZ-TAIC07}) has presented some approaches for addressing these issues.

\paragraph{Input arrays}
\mynote{Which paper???}It is common that programs may have loops bounded by the length of some input arrays. If this length cannot be statically determined then the symbolic engine may non-deterministically choose when to stop a loop. This means that the number of entries in a symbolic array does not need to be chosen as soon as the symbolic array is created, but this choice can be postponed.

%%% ENVIRONMENT

%In order to analyze a program, a symbolic executor has typically to take into account the whole environment surrounding it, including system libraries, devices etc.
%Environment can be seen as an input source. Since it can be unfeasible to analyze all possible interactions with the environment, it is common to model these interactions, emulating their behaviors and their side-effects. The main intuition is that models understand the semantics of the desired actions well enough to generate the required constraints.\\

\iffalse
%\cite{KLEE-OSDI08}: \begin{itemize}
%\item {\em file system}: operations on concrete files are actually performed. Operations on symbolic files are emulated modeling a simple symbolic file system, which is private for each execution state. Symbolic file system is a directory with $N$ symbolic files. Users specify both the number of files and their sizes. Any operation on a unconstrained symbolic file will generate $N+1$ branches: one for each symbolic file, plus one for a failing scenario. Emulation done at library level, not system call level. This make symbolic execution simpler (no need to symbolically execute library code) but assumption that library code is correct. If needed, library code is tested separately.
%\item {\em environment failures}: \cite{KLEE-OSDI08} emulates failures (e.g., failures of {\tt write}). This is optional since some applications may be not sensitive to environment failure.
 %\item {\em re-running test cases}: inputs which may crash an application may depend on the environment failures. To force concrete execution towards failures, \cite{KLEE-OSDI08} exploits {\tt ptrace}.
%\end{itemize}

\cite{AEG-NDSS11}:
\begin{itemize}
  \item {\em symbolic files}: emulation of {\tt open}, {\tt read}, {\tt write}, and similar other system calls. Similar to~\cite{KLEE-OSDI08}.
  \item {\em symbolic sockets}: emulation of {\tt socket}, {\tt bind}, {\tt accept}, {\tt send}, and similar other system calls. 
  \item {\em environment variables}: a complete summary of all possible results (concrete values, fully symbolic, and failures) of {\tt get\_env}.
  \item {\em library function calls and system calls}: emulation of more than 70 library routines and system calls. In particular, formatting functions (e.g., {\tt fprintf}) are emulated to capture buffer overflows.
\end{itemize}
\fi

%%% STATE MERGING
\paragraph{Selective state merging} Intermediate merging solutions can adopt heuristics for driving both merging decisions and CFG exploration. Generating larger symbolic expressions and possibly extra solvers invocations can outweigh the benefit of having fewer states, leading to poorer overall performance~\cite{HSS-RV09,KKB-PLDI12}. Moreover, in order to maximize the opportunities for merging a symbolic execution engine should traverse the CFG in a topological order, which prevents search exploration strategies aiming at prioritizing more ``interesting'' states over others. % denies search exploration
%(e.g., \cite{KKB-PLDI12} and \cite{VERITESTING-ICSE14}) 
Recent works have thus introduced novel techniques to tackle these issues: 
\begin{itemize}

\item {\em query count estimation}~\cite{KKB-PLDI12} identifies state merges that can reduce exploration time. This technique relies on a simple form of static analysis to identify how often each variable is used in branch conditions past any given point in the CFG. The estimate is used as a proxy for the number of solver queries that a given variable is likely to be part of. When two states are sufficiently similar, the overhead from solving more complex queries is likely to be outweighed by the savings from exploring fewer paths;

  \item {\em dynamic state merging}~\cite{KKB-PLDI12} efficiently combines static state merging with common search heuristics. This technique allows merging of states that do not share the same program location. This is useful, for instance, for unbounded loops for which search-based symbolic execution engines would employ search strategies that prioritize exploring new code over unrolling, while static state merging would require a depth-first exploration and thus fully unroll the possibly infinitely many iterations of the loop. Dynamic state merging can consider for merging states that are likely to become similar in a small number of execution steps: this is likely to happen if one state is similar to one of the predecessors of the other. The intuition behind the algorithm is that if two states are similar, then also their respective successors are likely to be similar after a few steps;

  \item {\em veritesting}~\cite{VERITESTING-ICSE14} dynamically identifies statements that generate formulas that are easy for a solver. Using a dynamically recovered CFG, it detects pieces of code that do not contain system calls, indirect jumps, or other statements that are difficult for static analysis. In particular, frontiers of hard-to-analyze statements are identified. Easy-to-analyze statements are analyzed maintaining a single formula describing all the merged states, while hard-to-analyze statements are evaluated using separate states as in traditional symbolic execution.
%Easy-to-analyze set of statements are then analyzed maintaining a single formula that describe all the merged states, while hard to analyze set of statements are evaluated using separate states pursuing the traditional symbolic execution approach.

\end{itemize}

%%% BINARY
\iffalse
\vspace{4em}
\mynote{Notes from D\&E start here} For software such as common off-the-shelf programs, neither users nor attackers have access to their source code: 

Challenges (e.g., ~\cite{BITBLAZE-ICISS08}):
\begin{enumerate}
\item Complexity of the instruction sets
\item Lack of a higher-level semantics (functions/CFG, types, buffers)
\item Obfuscation/dynamic code generation
\end{enumerate}

Symbolic techniques may work on the source code or on the binary code. However, it is not uncommon that both the former and the latter work by reasoning on an intermediate representation of the original code. For instance, ~\cite{KLEE-OSDI08} interprets the LLVM bytecode generated by compiling the source code, while~\cite{ANGR-SP16} reasons on the VEX IR that has been obtained by lifting the binary code.
\fi


%%% CONSTRAINTS

%Notice that the use of concrete values can also avoid to perform alias analysis on pointers, which is typically very expensive. Whenever meaningful, \cite{DART-PLDI05} tries to test both valid (not {\tt NULL}) and invalid ({\tt NULL}) input pointers in order to maximize bug detection. However~\cite{DART-PLDI05} will never artificially negate a branch if that condition cannot be exercised using a concrete input. In other words, both branches of an {\tt if} statement are considered only if they are both meaningful (more precisely: \cite{DART-PLDI05} is able to generate a valid input).

%Notice that~\cite{DART-PLDI05} may generate an input using a solver by considering only a subset of branch constraints. For instance, consider constraints ($C_1, C_2, C_3$) given by three nested branches: if $C_1$ is non linear (hard to solve), it needs only to generate a random input for taking $C_1$ and then use the solver for exploring path given by $(C_2, C_3)$. A traditional symbolic execution engine may get stuck at $C_1$ and give up after some time on {\em all} the derived path. Notice that whenever a concrete input is used to overcome a hard constraint, the overall approach become incomplete.

%~\cite{CKC-TOCS12} 
%Whenever a memory access with a symbolic pointer occurs,~\cite{CKC-TOCS12} determines the page referenced by the pointer. This information is then passed to the solver to help it. To make this even more effective, the page size is reduced as much as possible (e.g., 128 bytes).
%of constraints that might be hard or impossible to reason on for a solver. For instance, 

%A significant amount of the execution time of a symbolic engine is spent invoking the constraint solver.

\iffalse
\subsection{[OLD] Dealing with unsolvable constraints} 
\iffalse
Assume to start a concrete execution with a concrete input and in parallel symbolically execute the same program. Whenever a set of constraints cannot be solved by the constraint solver, then use the concrete value to proceed into at least one branch. Example taken from~\cite{CS-CACM13}:
    \begin{lstlisting}[basicstyle=\ttfamily\small]
    int non_linear(int v) {
      return (v * v) % 50;
    }
    \end{lstlisting}
The non-linear operation inside this function can be hard for a solver. Using a concrete execution, the engine can overcome this problem, but then the precision and completeness may be affected.
\fi

Notice that the use of concrete values can also avoid to perform alias analysis on pointers, which is typically very expensive. Whenever meaningful, \cite{DART-PLDI05} tries to test both valid (not {\tt NULL}) and invalid ({\tt NULL}) input pointers in order to maximize bug detection. However~\cite{DART-PLDI05} will never artificially negate a branch if that condition cannot be exercised using a concrete input. In other words, both branches of an {\tt if} statement are considered only if they are both meaningful (more precisely: \cite{DART-PLDI05} is able to generate a valid input). Notice that~\cite{DART-PLDI05} may generate an input using a solver by considering only a subset of branch constraints. For instance, consider constraints ($C_1, C_2, C_3$) given by three nested branches: if $C_1$ is non linear (hard to solve), it needs only to generate a random input for taking $C_1$ and then use the solver for exploring path given by $(C_2, C_3)$. A traditional symbolic execution engine may get stuck at $C_1$ and give up after some time on {\em all} the derived path. Notice that whenever a concrete input is used to overcome a hard constraint, the overall approach become incomplete.
\fi

% ---------------------------------------------------------------------------------------------------
\iffalse
\subsection{Solvers}
A list of constraint solvers\mynote{Table?}:
\begin{itemize}
  \item \cite{STP-TR07}: used by~\cite{EXE-CCS06,KLEE-OSDI08,MineSweeper-BOTNET08}, used by SPF
  \item \cite{Z3-TACS08}: used by~\cite{FIRMALICE-NDSS15,MAYHEM-SP12,SAGE-QUEUE12}
  \item \cite{DISSOLVER-TR03}: initially used by \cite{SAGE-NDSS08}
  \item \cite{PPL-SCP08}: used by \cite{CFB-ACSAC06}
  \item (incremental solver) \href{http://www.cs.nyu.edu/acsys/cvc3/}{CVC3}: an automatic theorem prover for Satisfiability Modulo Theories, used by SPF
  \item (incremental solver) \href{http://yices.csl.sri.com/}{Yices}: The Yices SMT Solver~\cite{YICES-CAV06}
  \item \href{http://choco-solver.org/}{CHOCO}: A Free and Open-Source Java Library for Constraint Programming, used by SPF
  \item \href{http://www.cs.brandeis.edu/~tim/Applets/IAsolver.html}{IAsolver}: the Brandeis Interval Arithmetic Constraint Solver, used by SPF
  \item \href{https://people.csail.mit.edu/akiezun/hampi/}{Hampi}: A Solver for String Constraints, used by SPF
  \item \href{https://www.cs.umd.edu/projects/omega/}{Omega}~\cite{OMEGA-SC91}, used by SPF
  \item \href{http://pagesperso.lina.univ-nantes.fr/~granvilliers-l/realpaver/}{RealPaver}: onlinear constraint solving \& rigorous global optimization, used by SPF
  \item \href{http://smtlib.cs.uiowa.edu/}{SMT-LIB}: the satisfiability modulo theories library, used by SPF
  \item CORAL~\cite{CORAL-NFM11}: used by SPF
\end{itemize}
\fi
% ---------------------------------------------------------------------------------------------------

% ---------------------------------------------------------------------------------------------------
\iffalse
\subsection{Lazy evaluation}\mynote{citations?}
\mynote{[D] original text starts here} A technique used in practice to avoid frequent invocation of the constraint solver is {\em lazy evaluation}. The main idea is to avoid checking for contradictions at each branch condition, exploring both branches. This means that some of the states that the engine will later explore may be non-reachable during a real program execution. This allows a symbolic engine to quickly explore paths but forces it to check their consistency before using them for drawing some conclusions. The main disadvantages of this techniques are: (a) this strategy may lead to a large number of active states, (b) the time spent for checking the consistency of many states may be similar to the time spent for performing pruning in the first place.
\fi

\iffalse

\subsubsection{Constraint caching} Cache solver results and reuse them \mynote{to be completed}

\subsubsection{Constraint independence} tracks constraints into multiple independent subsets of constraints, This helps the system discards irrelevant constraints and adds additional cache hits.  \mynote{to be completed}

\subsubsection{Irrelevant constraint optimization} Remove from path constraints those constraints that are irrelevant in deciding the outcome of the current branch. In practice, this is done by computing the transitive closure of all the constraints. Pointer and array reference can make this hard: e.g., see details in~\cite{EXE-CCS06,EGL-ISSTA09,CUTE-FSE13}. In~\cite{KLEE-OSDI08}, this is called {\em constraint independence}: the main idea is to divide constraints in independent disjoint subsets based on the symbolic variables which they reference. Irrelevant constraints can detected and discarded.

\subsubsection{Incremental solving} Many paths common branches, then it can be beneficial to reason about subset or superset of constraints. See more details, e.g., in~\cite{KLEE-OSDI08,CUTE-FSE13}. In~\cite{KLEE-OSDI08}, they propose {\em counterexample caching} to keep a cache of counterexample based on past queries:
      \begin{itemize}
        \item if a subset of constraints has not solution, any superset does not have as well
        \item if a superset has a solution, any subset has a solution
        \item if a subset has a solution, try it for the superset
      \end{itemize}

\subsubsection{Bit-field theory expression} \cite{CKC-TOCS12} tries to simplify expression using a bit-field theory expressions: if parts of a symbolic variable are masked away by bit operations, then known bits of the symbolic variable can be replaced by their constant values. If all bits of a symbolic variable are constant, then the variable is marked as concrete.

\fi

%%%%%% Under-constrained symbolic execution

\begin{comment}
\paragraph{Under-constrained KLEE}\mynote{I will revise this (emilio)} In~\cite{UCKLEE-USEC15}, KLEE~\cite{KLEE-OSDI08} has been extended in order to support under-constrained symbolic execution . In particular, these are some of the main improvements:
\begin{itemize}
  \item {\em Lazy initialization.} Whenever there is a pointer that is not concrete and without any active constraint (i.e., it is unbound), its value is checked against {\tt NULL} then two path are analyzed: (a) where the pointer is {\tt NULL} and (b) where the pointer is pointing to a freshly allocated block of memory, whose content is marked as unbound.This means that pointer aliasing is assumed to not occur. In both paths, the pointer is not longer unbound.  In path (b), any test against the pointer is known and dereferences will be successfully resolved. Lazy initialization is common for data structure and thus it is common to bound its maximum length (i.e., {\em k-bounding}) in order to prevent the engine from allocating an unbounded number of objects.
  \item {\em Patch checking.} The main goal of~\cite{UCKLEE-USEC15} is to detect if a patch has introduced new bugs. In order to do so, it symbolically executes two compiled versions of a function: $P$, the unpatched version, and $P'$, the patched version. If it finds any execution paths along which $P'$ crashes but $P$ does not (when given the samce symbolic inputs), it reports a potential bug. Indeed, due to missing input preconditions, not all crashes are real bugs: if both $P$ and $P'$ crash on an input, then maybe the crash is given by the unknown preconditions. 
  \item {\em Pruning techniques.} Using a static cross-checker (that navigates the control-flow graph and marks differing basic block between $P$ and $P'$), \cite{UCKLEE-USEC15} prunes paths that have never executed a differing basic block and that cannot reach a differing basic block from their current program counter and call stack. Moreover, $P'$ is executed before executing $P$, allowing the system to prune paths that return from $P'$ without triggering an error, or that trigger an error without reaching different blocks.
  \item {Dealing with false positives.} Two approaches are pursued to limit false positives:
    \begin{itemize}
      \item {\em manual annotations}: examples are data types invariants or preconditions upon function calls.
      \item {\em automated heuristics}: {\em must-fail} heuristics identify errors that must occur for all input values following that execution path. For instance, {\em belief-fail} heuristic checks if a function contradicts itself (e.g., a code checks that a pointer is {\tt NULL} and then dereferences it). Another variation of must-fail heuristic is {\em concrete-fail} that an assertion failure or memory errors was triggered by a concrete condition or pointer.
    \end{itemize}
  \item {\em Rule-based checkers.} Several rule-based checkers have been built on top of UCKLEE. They do a similar job such as other dynamic tools (e.g., {\em memcheck} for memory leaks and uninitialized data) but reasoning on all possible paths, not just the concrete ones. Moreover, user inputs can be considered as fully constrained (i.e., no assumption is valid on it since the code should sanitize it). 
  \item {\em Optimizations.} Some optimizations:
    \begin{itemize}
      \item Symbolic objects have a symbolic size: whenever there is an access to the object content, the system verifies if the offset could exceed the object's symbolic size. whenever path is considered where the offset does not exceed the symbolic size, then a lower bound on the symbolic size is set.
      \item Some library functions (e.g., {\tt strlen}) have been replaced with variants that do not lead to path explosion
      \item Scores of rules to simplify symbolic expressions
      \item {\em lazy constraints}: defer evaluation of constraints using a solver as much as possible. For instance, if there is an hard constraint on branch, take both branches and if an error is found check if that branch was actually valid.
      \item function pointers should be made concrete by the user.
    \end{itemize}

\end{itemize}
\end{comment}


% ---------------------------------------------------------------------------------------------------
\subsection{Constraint optimizations}
\label{constraint-optimizations}

\mynote{add \cite{S-FMCAD08} + some optimizations listed in Section~\ref{memory-model}.}

Many optimization can be applied to constraints in order to make it more solver-friendly.

\mynote{[D] TOCS page 24}\subsubsection{Page boundings on pointers} Whenever a memory access with a symbolic pointer occurs,~\cite{CKC-TOCS12} determines the page referenced by the pointer. This information is then passed to the solver to help it. To make this even more effective, the page size is reduced as much as possible (e.g., 128 bytes).

%{\em Concolic execution} has been originally introduced in~\cite{DART-PLDI05} and then refined by~\cite{CUTE-FSE13}. A common disadvantage of symbolic execution is that the state space can be exponential. Moreover, even when the state space is tractable it may happen that complex constraints need to be solved but these constraints are too complex for the actual solver (e.g., non-linear constraints are typically hard to solve). For this reason, it is common to exploit concolic execution. 