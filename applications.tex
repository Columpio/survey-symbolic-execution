% !TEX root = main.tex

\section{Applications}\mynote{Case studies?}

\subsection{Authentication bypass}
\cite{FIRMALICE-NDSS15}

\subsection{Bug detection}
\cite{DRILLER-NDSS16}

\subsection{Bug exploitation}
\cite{MAYHEM-SP12}

\subsection{Tools}
A list of symbolic execution engines is presented in Table~\ref{tab:symbolic-engines}.

\begin{figure}[t]
  \centering
  \begin{adjustbox}{width=1\columnwidth}
  \begin{small}
  \begin{tabular}{| l || c || l |}
    \hline      
    Symbolic engine & Paper(s) & Project URL  \\ \hline\hline
    {\tt angr} & \cite{FIRMALICE-NDSS15} & \url{http://angr.io/} \\
    {\tt KLEE}: a LLVM Execution Engine & \cite{KLEE-OSDI08} & \url{https://klee.github.io/} \\
    {\tt jCUTE}: Java Concolic Unit Testing Engine & -- & \url{https://github.com/osl/jcute} \\
    {\tt Java PathFinder} & \cite{PATHFINDER-ASE10} & \url{http://babelfish.arc.nasa.gov/trac/jpf}\\
    {\tt CIVIL}: The Concurrency Intermediate Verification Language & add technical report & \url{http://vsl.cis.udel.edu/civl/}\\
    {\tt CREST}: a concolic test generation tool for C & \cite{CREST-ASE08} & \url{https://github.com/jburnim/crest} \\
    {\tt PEX} & \cite{PEX-TAP08} & \url{http://research.microsoft.com/en-us/projects/pex/} \\
    {\tt Check 'n' Crash} & \cite{CS-ICSE05} & \url{http://ranger.uta.edu/~csallner/cnc/}\\
    \hline  
  \end{tabular}
  \end{small}
  \end{adjustbox}
  \caption{List of tools.}
  \label{tab:symbolic-engines}
\end{figure}