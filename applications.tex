% !TEX root = main.tex

\section{Sample Applications}
\label{se:applications}

\subsection{Bug Detection}
\label{ss:bug-detection}

Software testing strategies typical attempt to execute a program with the intent of finding bugs. As manual test input generation is an error-prone and usually not exhaustive process, automated testing technique have drawn a lot of attention over the years. Random testing techniques such as fuzzing are cheap in terms of run-time overhead, but fail to obtain a wide exploration of a program state space. Symbolic and concolic execution techniques on the other hand achieve a more exhaustive exploration, but they become expensive as the length of the execution grows: for this reason, they usually reveal shallow bugs only.

\cite{RK-ICSE07} proposes {\em hybrid concolic testing} for test input generation, which combines the ability of random search to reach deep program states with the ability of concolic execution to achieve a wide exploration. The two techniques are interleaved: in particular, when random testing saturates (i.e., it is unable to hit new code coverage points after a number of steps), concolic execution is used to mutate the current program state by performing a bounded depth search for an uncovered coverage point. For a fixed time budget, the technique outperforms both random and concolic testing in terms of branch coverage. The intuition behind this approach is that many programs show behaviors where a state can be easily reached through random testing, but then a precise sequence of events - identifiable by a symbolic engine - is required to hit a specific coverage point.

\cite{DRILLER-NDSS16} refines this idea and devises a vulnerability excavation tool based on {\sc Angr}~\cite{ANGR-SSP16}, called {\em Driller}, that interleaves fuzzing and concolic execution to discover memory corruption vulnerabilities. The authors remark that user inputs can be categorized as {\em general}  input, which has a wide range of valid values, and {\em specific} input: a check for particular values of a specific input then splits an application into {\em compartments}. {\em Driller} offloads the majority of unique path discovery to a fuzzy engine, and relies on concolic execution to move across compartments. During the fuzzy phase, {\em Driller} marks a number of inputs as interesting (for instance, when it was the first to trigger some state transition) and once it gets stuck in the exploration, it passes the set of such paths to a concolic engine, which uses preconstraining on the program states to ensure consistency with the results of the native execution. {\sc Angr}, and thus {\em Driller}, uses an index-based memory model as in Section~\ref{ss:index-based-memory} where reads can be symbolic and writes are always concretized. % read/write addresses

\subsection{Authentication Bypass}
\label{ss:auth-bypass}
Software backdoors are a method of bypassing authentication in an algorithm, a software product, or even in a full computer system. Although sometimes these software flaws are injected by external attackers using subtle tricks such as compiler tampering~\cite{KRS-TR74}, there are reported cases of backdoors that have been surreptitiously installed by the hardware and/or software manufacturers~\cite{CZF-USEC14}, or even by governments~\cite{NSA-BACKDOOR}. 

Different works~\cite{ZBF-NDSS14,DMR-USEC13,FIRMALICE-NDSS15} have exploited symbolic execution for analyzing the behavior of binary firmwares. Indeed, an advantage of this technique is that it can used even in environments, such as embedded systems, where the documentation and the source code that are publicly released by the manufacturer are typically very limited or none at all. For instance,~\cite{FIRMALICE-NDSS15} proposes {\em Firmalice}, a binary analysis framework based on {\sc Angr}~\cite{ANGR-SSP16} that can be effectively used for identifying authentication bypass flaws inside firmwares running on devices such as routers and printers. Given a user-provided description of a privileged operation in the device, {\em Firmalice} identifies a set of program points that, if executed, forces the privileged operation to be performed. Exploiting {\sc Angr}, the program slice that involves the privileged program points is symbolically analyzed. If any of the program points can be reached by the engine, a set of concrete inputs is generated using a SMT solver. These values can be then used to effectively bypass authentication inside the device.

\subsection{Bug Exploitation}
\label{ss:bug-exploitation}
Bugs are a consequence of the nature of human factors in software development and are everywhere. Those can be exploited by an attacker should normally be fixed first, thus systems to automatically and effectively identify them are very valuable.

{\sc AEG}~\cite{AEG-NDSS11} employs preconditioned symbolic execution to analyze a potentially buggy program in source from and look for bugs amenable to a return-to-stack or return-to-libc exploits, which are popular control hijack attack techniques. The tool augments path constraints with exploitability constraints and queries a constraint solver, generating a concrete exploit when the constraints are satisfiable. The authors devise the {\em buggy-path-first} and {\em loop-exhaustion} strategies discussed in Section~\ref{sss:search-heuristics} to prioritize paths in the search. 

{\sc Mayhem}~\cite{MAYHEM-SP12} takes another step forward by presenting the first end-to-end exploitable bug finding system working on programs in binary form. It adopts a hybrid execution model based on checkpoints and two components: a concrete executor that injects taint-analysis instrumentation in the code and a symbolic executor that takes over when a tainted branch or jump instruction is met. Exploitability constraints for symbolic instruction pointers and format strings are generated, targeting a wide range of exploits, e.g., SEH-based and jump-to-register ones. Three path selection heuristics help prioritizing paths that are most likely to contain a bug (e.g., those containing symbolic memory accesses or instruction pointers). A virtualization layer intercepts and emulates all the system calls to the host OS, while preconditioned symbolic execution can be used to reduce the range of the search space. Also, restricting symbolic execution to tainted basic blocks only gives very good speedups in this setting, as in the reported experiments more than $95\%$ of the instructions were not tainted. Although the goal in {\sc Mayhem}~\cite{MAYHEM-SP12} is informing the user that an exploitable bug exists, the generated simple exploits can be likely transformed in an automated fashion to work in the presence of OS defenses such as address space layout randomization and data execution prevention~\cite{Q-SEC11}. 

